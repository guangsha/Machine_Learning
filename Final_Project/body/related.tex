\section{Related Works}
Besides CWMR, a few other mean-reversion based portfolio selection strategies\cite{UniversalPortfolio, MAFI:MAFI274} have been reported since the idea of mean reversion was first proposed in 1990\cite{mean_reverse}. One of them is the Anticor algorithm introduced by Borodin \emph{et al.} in 2004\cite{AntiCor}.

As we discussed in the previous section, CWMR updates the portfolio for the next trading day by choosing one that has a poor performance today but does not deviate too much from the portfolio used for today. Anticor algorithm has similar characters since it also believes that poor-performing stock is incline to reverse, but it is also very different from CWMR in that Anticor focuses more on the correlations among the stocks.

While CWMR considers the evolution of portfolio as a whole, Anticor applies the mean reversion theory at a much smaller scale, and more specifically, it measures the shift of investment between each pair of stocks every time the portfolio is updated.

The Anticor algorithm evaluates variations in stocks’ performance by dividing a sequence of previous trading days into two periods called windows, with equal length of $w$ days. Anticor tends to transfer investment from stock $i$ to stock $j$ when the following conditions are fulfilled:

1. Stock $i$ had a better performance than stock $j$ over the most recent window.

2. Stock $i$ over the second most recent window is positively correlated with stock $j$ over the most recent window.

3. Both stock $i$ and $j$ are anti-correlated with themselves over consecutive windows.

Figure~\ref{fig:illustration}(b) illustrates a typical case when the investment will be shifted from stock $i$ to $j$. Since the correlation of between stocks $i$ and $j$ in consecutive windows and the anti-correlation of both stocks are supposed to be maintained in the next window, both stocks tend to reverse in the window. Stock $j$, which did not perform well in the most recent window, is predicted to rise in the next window.

For each pair of stocks $i$ and $j$, the extent to which the investment should be shifted from stock $i$ to stock $j$ in the next trading is quantified by $claim_{i\rightarrow j} = M_{cor}(i, j) - M_{cor}(i, i) - M_{cor}(j, j)$ where $M_{cor}(i, j)$ is the correlation between stock $i$ over the second most recent window and stock $j$ over the most recent window, and $M_{cor}(i, i)$ is the correlation between stock $i$ over the second most recent window and itself over the most recent window. Therefore, $claim_{i\rightarrow j}$ will be large if 1) stock $i$ over the second most recent window is positively correlated with stock $j$ over the most recent window and 2) both stock $i$ and $j$ are anti-correlated with themselves over consecutive windows.

As we will show in the Evaluation section, the Anticor algorithm also results in a huge cumulative wealth by constantly transferring investment from the well-performing stock to the anti-correlated poorly-performing stock. The major problem with Anticor is that the return is very sensitive to the window size $w$, the optimal window size may range from 10 days to 50 days depending on which period of time we are interested in.
