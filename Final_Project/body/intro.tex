\section{Introduction}
Online portfolio selection is a sequential process, which aims to determine a practical strategy for allocating investments among a set of stocks to achieve some financial goals in the long run. Most portfolio selection strategies can be classified into two large groups: trend following and mean reversion. The trend following strategy assumes that historically better-performing stocks would also perform better than others in future. Although this idea is easy to understand, it was found to be violated in short term, for example, if we try to follow the daily trend\cite{JOFI:JOFI5110}.

The mean reversion strategy, on the other hand, believes that if a stock performs worse than others today, it tends to perform better than others in the next trading day. One of the most robust mean reversion methods is the Anticor algorithm proposed by Borodin \emph{et al.}, which obtains a high profit by actively reverting to the mean\cite{AntiCor}.

The CWMR strategy we focus on in the report has an even better performance than Anticor, because it does not only exploit the first order information of the portfolio (mean), but also reflects the second order information (volaility) by constructing the Gaussian distribution of the portfolio vector\cite{OnlinePortfolio}. The key idea of CWMR is to formulate the Gaussian distribution of the portfolio vector that can effectively exploit both its first and second order information and sequentially update this distribution for every trading day. When the portfolio is updated, the CWMR strategy either passively keeps last portfolio, if it had a poor performance in the previous day, or aggressively approaches a new portfolio that would have had an even worse performance in the previous day.

With the CWMR strategy, Li \emph{et al.} reported a cumulative gain of around 10$^{18}$\% after 5600 trading days with a portfolio including 36 NYSE stocks. In this project we implemented the CWMR method and successfully reproduced this astronomical value with the same dataset as they used. To investigate the reliability of CWMR method, we tested it on the up-to-date stock dataset, and found this huge gain was also achievable in different time periods and various stock markets. To better understand the advantage of CWMR method, we also studied and implemented the state-of-art Anticor method for comparison. Finally, we proposed a revised method, named ``A-Stock-A-Day Confidence Weighted Mean Reversion" (ASAD-CWMR), which trades only one stock instead of tens or hundreds of them everyday. ASAD-CWMR method greatly reduces the transaction cost but still maintain the power of mean reversion and guarantee a huge gain in certain periods of time.

