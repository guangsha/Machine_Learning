\section{Overview}
We consider a financial market with $m$ stocks to invest in for $n$ trading days.  The change of stock prices everyday is represented by a price relative vector $\textbf{x} \in \textbf{R}^m$ with each component denoting the ratio of the closing price to the last closing price of one of the m stocks. A sequence of these price relative vectors $\textbf{x}_1, \cdots, \textbf{x}_n$ describes the price change of the m stocks during n trading days. More specifically, if we invest in the $j$th stock, we will have a reward of $x_{ij}$ on the $i$th trading day.

An investment on the market is specified by a portfolio vector, denoted as $\textbf{b}=(b_1,\cdots,b_m)$, where $b_i$ is the proportion of wealth invested on the $i$th asset. The feasible space of $\textbf{b}$ is denoted by $\Delta_m$, where $\Delta_m =\{\textbf{b}:\textbf{b} \in \textbf{R}^m, \sum_{i=1}^m b_i=1\}$. The investment on the $i$th trading day is defined by a portfolio $\textbf{b}_i$, which increases the wealth by a factor $s_i = \textbf{b}_i^{T}x_i = \sum_{j=1}^m b_{ij}x_{ij}$.  $s_i$ is the portfolio daily return. $\textbf{b}^n$ denotes the portfolio strategy for the $n$ consecutive trading days. Since we re-invest all the wealth, the investment results in multiplicative cumulative return. Thus, after $n$ trading days, the investment of a portfolio strategy $\textbf{b}^n$ produces a portfolio cumulative wealth $\textbf{S}_n$, which is increased by a factor of $\prod_{i=1}^n\textbf{b}_i^T x_i$, i.e., $\textbf{S}_n(\textbf{b}^n,\textbf{x}^n) = \textbf{S}_0\prod_{i=1}^n\textbf{b}_i^T x_i$, where $\textbf{S}_0$ denotes the initial wealth, which is set to \$1 here.

The online portfolio selection problem is a sequential decision task, and we aim to design a strategy to maximize the portfolio cumulative wealth in a sequential fashion. On each trading period $i$, given the historical information, including all the previous sequences of price relative vectors $\textbf{x}^{i-1} =\{x_1 , . . . , x_{i-1} \}$, and the previous sequences of portfolio vectors $\textbf{b}^{i-1} =\{b_1 , \cdots, b_{i-1}\}$, we decide a new portfolio vector $\textbf{b}^i$ for the coming price relative vector $\textbf{x}_i$. This learning and predicting process repeats until the end of the trading period, e.g., $n$ trading days. A portfolio stratergy is evaluated based on the cumulative wealth achieved in the end.

Most portfolio selection strategies, including the methods we will introduce in this report, are based on the three assumptions including (1) no transaction cost; (2) each stock is arbitrarily divisible tradable at its closing price; (3) our investment has no significant impact on the market behavior.

%\subsection{Formulation}


